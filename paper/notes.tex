\documentclass{article}
\usepackage{amsmath}
\usepackage{amssymb}
\usepackage{mathpartir}
\usepackage{mathtools}

\usepackage[T1]{fontenc}
\usepackage{libertine}
\usepackage{url}
\usepackage[scaled=0.83]{beramono}
\newcommand\Grit{\texttt{grit}}
\newcommand\TinyLF{TinyLF}

\renewcommand\vec[1]{\overline{#1}}


%%% PLEASE FEEL FREE TO CHANGE NOTATION!
\newcommand\SORT{\blacksquare}
\newcommand\Ar[2]{[#1]#2}
\newcommand\TlNil{\cdot}
\newcommand\TlSnoc[3]{#1,#2:#3}
\newcommand\SpNil{\cdot}
\newcommand\SpSnoc[2]{#1,#2}
\newcommand\Ap[2]{#1[#2]}
\newcommand\Abs[2]{[#1]#2}

\newcommand\WfSrt[2]{{#1}\vdash{#2}\;\textit{sort}}
\newcommand\WfAr[2]{{#1}\vdash{#2}\;\textit{arity}}
\newcommand\WfTl[2]{{#1}\vdash{#2}\;\textit{tele}}
\newcommand\WfTm[3]{{#1}\vdash{#2}\Rightarrow{#3}}
\newcommand\WfSp[3]{{#1}\vdash{#2}\Leftarrow{#3}}
\newcommand\WfBnd[3]{{#1}\vdash{#2}\Leftarrow{#3}}
\newcommand\HSubst[3]{[#1/#2]#3}

\title{\Grit: the kernel around which a PRL forms}
\author{}
\date{}

\begin{document}
\maketitle

\section{Syntactic Framework}

\Grit{} is based on dependently-sorted higher-order abstract syntax,
inspired by Reed's \TinyLF{}~\cite{reed:2007} and Fiore's
dependently-sorted syntax~\cite{fiore:2008}. The grammar is given as
follows:

\[
  \begin{array}{llll}
    \text{sorts}
    &\tau
    &::=
    &\SORT\mid R
    \\
    \text{arities}
    &\alpha
    &::=
    &\Ar{\Psi}{\tau}
    \\
    \text{telescopes}
    &\Psi
    &::=
    &\TlNil\mid\TlSnoc{\Psi}{x}{\alpha}
    \\
    \text{terms}
    &R
    &::=
    &\Ap{x}{\psi}
    \\
    \text{binders}
    &M
    &::=
    &\Abs{\vec{x}}{R}
    \\
    \text{spines}
    &\psi
    &::=
    &\cdot\mid\SpSnoc{\psi}{M}
  \end{array}
\]

The wellformedness conditions for each of these forms of syntax are
given together with hereditary substitutions by a simultaneous
inductive-recursive definition.
\begin{mathpar}
  \inferrule{
  }{
    \WfSrt{\Psi}{\SORT}
  }
  \and
  \inferrule{
    \Psi(x) = \Ar{\Phi}{\SORT}
    \\
    \WfSp{\Psi}{\phi}{\Phi}
  }{
    \WfSrt{\Psi}{\Ap{x}{\phi}}
  }
  \and
  \inferrule{
    \WfTl{\Psi}{\Phi}
    \\
    \WfSrt{\Psi,\Phi}{\tau}
  }{
    \WfAr{\Psi}{\Ar{\Phi}{\tau}}
  }
  \and
  \inferrule{
  }{
    \WfTl{\Psi}{\TlNil}
  }
  \and
  \inferrule{
    \WfTl{\Psi}{\Phi}
    \\
    \WfAr{\Psi,\Phi}{\alpha}
  }{
    \WfTl{\Psi}{\TlSnoc{\Phi}{x}{\alpha}}
  }
  \and
  \inferrule{
    \Psi(x) = \Ar{\Phi}{\tau}
    \\
    \WfSp{\Psi}{\phi}{\Phi}
  }{
    \WfTm{\Psi}{\Ap{x}{\phi}}{\HSubst{\psi}{\Phi}{\tau}}
  }
  \and
  \inferrule{
    \WfTm{\Psi,\vec{x_i:\alpha_i}}{R}{\tau}
  }{
    \WfBnd{\Psi}{\Abs{\vec{x_i}}{R}}{\Ar{\vec{x_i:\alpha_i}}{\tau}}
  }
  \and
  \inferrule{
  }{
    \WfSp{\Psi}{\SpNil}{\TlNil}
  }
  \and
  \inferrule{
    \WfSp{\Psi}{\phi}{\Phi}
    \\
    \WfBnd{\Psi}{M}{\HSubst{\phi}{\Phi}{\alpha}}
  }{
    \WfSp{\Psi}{\SpSnoc{\phi}{M}}{\TlSnoc{\Phi}{x}{\alpha}}
  }
\end{mathpar}

\begin{align*}
  \HSubst{\psi}{\Psi}{\SORT}
  &=
  \SORT
  \\
  \HSubst{\psi}{\Psi}{(\Ap{x}{\phi})}
  &=
  \begin{cases*}
    \HSubst{
      \HSubst{\psi}{\Psi}{\phi}
    }{\vec{y_i:\alpha_i}}{R} 
    &if $\Psi(x)=\Ar{\vec{y_i:\alpha_i}}{\tau}, \psi(x)=\Abs{\vec{y_i}}{R}$
    \\
    \Ap{x}{
      \HSubst{\psi}{\Psi}{\phi}
    }
    &otherwise
  \end{cases*}
  \\
  \HSubst{\psi}{\Psi}{(\TlNil)}
  &=
  \TlNil
  \\
  \HSubst{\psi}{\Psi}{(\TlSnoc{\vec{x_i:\alpha_i}}{y}{\beta})}
  &=
  \TlSnoc{
    \HSubst{\psi}{\Psi}{(\vec{x_i:\alpha_i})}
  }{x}{
    \HSubst{\psi}{\Psi}{\beta}
  }
  \\
  \HSubst{\psi}{\Psi}{\Abs{\vec{x_i}}{R}}
  &=
  \Abs{\vec{x_i}}{
    (\HSubst{\psi}{\Psi}{R})
  }
  \\
  \HSubst{\psi}{\Psi}{(\SpNil)}
  &=
  \SpNil
  \\
  \HSubst{\psi}{\Psi}{(\SpSnoc{\phi}{M})}
  &=
  \SpSnoc{\HSubst{\psi}{\Psi}{\phi}}{\HSubst{\psi}{\Psi}{M}}
\end{align*}

\section{Refinement Logic}

\newcommand\Spec[2]{{#1}\triangleright{#2}}
\newcommand\SigNil{\cdot}
\newcommand\SigSnoc[3]{#1,#2:#3}

\newcommand\Seq[2]{#1\gg{#2}}

\newcommand\RfSig[2]{{#1}\vdash{#2}\;\textit{sig}}
\newcommand\RfSpec[2]{{#1}\vdash{#2}\;\textit{spec}}
\newcommand\RfSrt[4]{#1\mathrel{\vdash_{#2}}{#3}\sqsubset{#4}}
\newcommand\RfAr[4]{#1\mathrel{\vdash_{#2}}{#3}\sqsubset{#4}}
\newcommand\RfTl[4]{#1\mathrel{\vdash_{#2}}{#3}\sqsubset{#4}}

On top of the syntactic framework is layered a notion of
\emph{refinement judgment}, organized around refinement signatures
$\vec{\vartheta_i:\Spec{\Psi_i}{\tau_i}}$; in such a signature,
$\vartheta_i$ is the name of an atomic form of judgment which takes
inputs of arities $\Psi_i$ and has an output (synthesis) of sort
$\tau_i$.

\[
  \begin{array}{llll}
    \text{specifications}
    &\mathfrak{s}
    &::=
    &\Spec{\Psi}{\tau}
    \\
    \text{signatures}
    &\Sigma
    &::=
    &\SigNil\mid\SigSnoc{\Sigma}{\vartheta}{\mathfrak{s}}
    \\
    \text{judgments}
    &\mathcal{J}
    &::=
    &\Ap{\vartheta}{\psi}
    \\
    \text{sequents}
    &\mathcal{S}
    &::=
    &\Seq{\mathcal{H}}{\mathcal{J}}
    \\
    \text{telescopes}
    &\mathcal{H}
    &::=
    &\TlNil\mid\TlSnoc{\mathcal{H}}{x}{\mathcal{S}}
  \end{array}
\]

Below, we specify the wellformedness of refinement signatures and
refinements simultaneously with their erasure. This may not be how it
ends up being \emph{implemented}, but it is the most natural way to
explain it on paper.

\begin{mathpar}
  \inferrule[specification]{
    \WfTl{\Psi}{\Phi}
    \\
    \WfSrt{\Psi,\Phi}{\tau}
  }{
    \RfSpec{\Psi}{\Spec{\Phi}{\tau}}
  }
  \and
  \inferrule[sig/nil]{
  }{
    \RfSig{\Psi}{\SigNil}
  }
  \and
  \inferrule[sig/snoc]{
    \RfSig{\Psi}{\Sigma}
    \\
    \RfSpec{\Psi}{\mathfrak{s}}
  }{
    \RfSig{\Psi}{\SigSnoc{\Sigma}{\vartheta}{\mathfrak{s}}}
  }
  \and
  \inferrule[atomic judgment]{
    \Sigma(\vartheta) = \Spec{\Phi}{\tau}
    \\
    \WfSp{\Psi}{\phi}{\Phi}
  }{
    \RfSrt{\Psi}{\Sigma}{\Ap{\vartheta}{\phi}}{
      \HSubst{\phi}{\Phi}{\tau}
    }
  }
  \and
  \inferrule[sequent]{
    \RfTl{\Psi}{\Sigma}{\mathcal{H}}{\Phi}
    \\
    \RfTl{\Psi,\Phi}{\Sigma}{\mathcal{J}}{\tau}
  }{
    \RfAr{\Psi}{\Sigma}{\Seq{\mathcal{H}}{\mathcal{J}}}{\Ar{\Phi}{\tau}}
  }
  \and
  \inferrule[tele/nil]{
  }{
    \RfTl{\Psi}{\Sigma}{\TlNil}{\TlNil}
  }
  \and
  \inferrule[tele/snoc]{
    \RfTl{\Psi}{\Sigma}{\mathcal{H}}{\Phi}
    \\
    \RfAr{\Psi,\Phi}{\Sigma}{\mathcal{S}}{\alpha}
  }{
    \RfTl{\Psi}{\Sigma}{
      (\TlSnoc{\mathcal{H}}{x}{\mathcal{S}})
    }{
      (\TlSnoc{\Phi}{x}{\alpha})
    }
  }
\end{mathpar}

\subsection{Proof States and Refinement Rules}
\newcommand\State[3]{#1\blacktriangleright{#2}:{#3}}
% \newcommand\Rule[3]{#1\Longrightarrow{#2}\blacktriangleright{#3}}
\newcommand\WfState[3]{#1\mathrel{\vdash_{#2}}#3\;\textit{state}}

We extend the grammar with \emph{proof states} below.
\[
  \begin{array}{llll}
    \text{states}
    &\mathcal{C}
    &::=
    &\State{\mathcal{H}}{M}{\alpha}
    % \\
    % \text{rule}
    % &\mathcal{R}
    % &::=
    % &\Rule{\mathcal{S}}{\mathcal{H}}{M}
  \end{array}
\]

The data of a proof state $\State{\mathcal{H}}{M}{\alpha}$ is a
refinement telescope of \emph{goals} $\mathcal{H}$ together with a
piece of partial evidence $M$ of arity $\alpha$. Note that $\alpha$ is
an ordinary arity, rather than a sequent: in the proof refinement
process, it is never necessary to know the ``main goal''.

The wellformedness conditions for proof states are simple:
\begin{mathpar}
  \inferrule{
    \RfAr{\Psi}{\Sigma}{\mathcal{H}}{\Phi}
    \\
    \WfAr{\Psi,\Phi}{\alpha}
    \\
    \WfBnd{\Psi,\Phi}{M}{\alpha}
  }{
    \WfState{\Psi}{\Sigma}{
      (\State{\mathcal{H}}{M}{\alpha})
    }
  }
\end{mathpar}

% TODO: discuss syntax of refinement rules; to do this, we need a
% well-defined notion of pattern. This could be accomplished by adding
% an extra context of \emph{linear} variables to the judgments for
% refinements.


\bibliographystyle{plain}
\bibliography{refs}

\end{document}
